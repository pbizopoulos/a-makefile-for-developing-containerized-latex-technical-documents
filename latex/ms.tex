\documentclass[journal]{IEEEtran}
\usepackage{dirtree}
\usepackage[boxed]{algorithm2e}
\usepackage[nopdftrailerid=true]{pdfprivacy}
\usepackage{hyperref}

\begin{document}

\title{Abstraction-averse programming}

\author{Paschalis~Bizopoulos
\thanks{P. Bizopoulos is an Independent Researcher, Thessaloniki, Greece e-mail: pbizop@gmail.com}}

\maketitle

\begin{abstract}
	We propose a software development methodology for portable and consistent code and also refer to example use cases.
\end{abstract}

\section{Introduction}
The most important problem today with software is its increasing complexity which can be attributed to the use of unnessesary abstractions.
Abstractions can be variables, functions, classes, modules, files, directories, third-party dependencies.
Most of the times using already existing abstractions is better than creating/introducing new ones.

Code is usually targeted to be executed by multiple machines and developed by multiple humans.
Therefore, it is vital to have a portable way for machines to execute code and a consistent way for humans to develop code.
In this paper we present a simple software development methodology which satisfies these conditions.
We propose using POSIX XCU and Containers for code portability and a directory structure enforcement for code consistency.
We also refer example use cases for python machine learning applications, \LaTeX{} documents generation, web applications and python packages.

\section{Abstraction-averse programming}
The phrase "Abstraction-averse" is in contrast to "Abstraction-free" where would need to program in machine language (0s and 1s) exactly as the CPU sees them.
However "Abstraction-averse" recognizes the practical aspect of abstraction but instructs that the introduction/creation of abstractions should be avoided as much as possible.

\section{Code portability}
POSIX (Portable Operating System Interface) is a set of standards developed by the Austin Group with the target of maintaining compatibility between operating systems.
More specifically the XCU (Shell and Utilities) volume of POSIX focuses on defining specifications for common utility applications.
However, POSIX XCU is not enough to cover modern user/developer requirements, such as machine learning and web applications.

Containers such as Docker, can be used to create isolated environments.
Containers were standardized with the Open Container Initiative\footnote{\url{https://opencontainers.org/}} and there exist alternatives such as Podman that could be used as drop-in replacements for Docker.

\section{Code Consistency}
On a higher level the directory should look like this:
\dirtree{%
.1 \texttt{\$HOME/}.
.2 \texttt{\$PLATFORM\_DOMAIN\_NAME/}.
.3 \texttt{\$REPOSITORY\_OWNER/}.
.4 \texttt{\$REPOSITORY\_NAME/}.
}

For example the directory structure that corresponds to the repository that hosts the source code of this paper is:
\dirtree{%
.1 \texttt{\$HOME/}.
.2 \texttt{github.com/}.
.3 \texttt{pbizopoulos/}.
.4 \texttt{software-development-methodology/}.
}

The following directory structure is the superset of files and directories that the \texttt{python/} directory can consist.
File and directory name whitelisting is enforced with the target of having one file per `technology'.
\dirtree{%
.1 python/|docs/|latex/.
.2 \texttt{.dockerignore}.
.2 \texttt{.gitignore}.
.2 Dockerfile.
.2 Makefile.
.2 main.py|index.html|ms.tex.
.2 prm/.
.2 pyproject.toml|package.json.
.2 tmp/.
}

More specifically:
\begin{itemize}
	\item \texttt{.dockerignore}: Consists of:
		\begin{itemize}
			\item * (every file/directory)
			\item except files needed for environment creation (e.g.\ \texttt{pyproject.toml})
		\end{itemize}
	\item \texttt{.gitignore}: Consists of:
		\begin{itemize}
			\item the temporary directory \texttt{tmp/} and
			\item the environmental variable file \texttt{.env}.
		\end{itemize}
	\item \texttt{Dockerfile}: Rarely editable. The developer edits:
		\begin{itemize}
			\item the FROM image and
			\item whether other apt-get installations are done.
		\end{itemize}
	\item \texttt{Makefile}: Non-editable. Uses a subset of the GNU standard targets. Properties~\cite{stallman1992gnu}:
		\begin{itemize}
			\item It should be POSIX-compliant and use POSIX utilities~\cite{lewine1991posix}.
			\item The \texttt{all} target should generate data saved in \texttt{tmp/}.
			\item The \texttt{check} target should check all \texttt{code}. It should also check and disallow any additional files or directories besides the ones defined here.
			\item The \texttt{clean} target should delete \texttt{tmp/}.
			\item It should support a \texttt{STAGING} environment variable which can be used for debugging purposes and having fast development iteration.
			\item It should create all files and directories from the structure if they do not already exist.
		\end{itemize}
	\item \texttt{main.py}: Frequently editable. The developer writes \texttt{code} in here that generates data saved in \texttt{tmp/}.
	\item \texttt{prm/}: The developer can place here all the files/directories for which generation from code is infeasible. It is a practical `backdoor' out of the file/directory name whitelisting of the Makefile. The contents of this directory should be as few as possible.
	\item \texttt{pyproject.toml}: Rarely editable. The developer edits:
		\begin{itemize}
			\item the dependencies and
			\item the version of the dependencies
		\end{itemize}
	\item \texttt{tmp/}: Data generated by \texttt{main.py} are saved here.
\end{itemize}

Code consistency on the directory structure level also brings code portability and also easier execution of commands on multiple repositories.
Using POSIX Utilities we can use the following:
\begin{verbatim}
find $HOME/$DIRECTORY -name $DIRECTORY_OR_FILE |\
xargs dirname |\
xargs -I{} -P0 $COMMAND
\end{verbatim}

For example the following incorporates changes from all \texttt{github.com} remote repositories into the current branches using \texttt{git}:
\begin{verbatim}
find $HOME/github.com -name .git |\
xargs dirname |\
xargs -I{} -P0 git -C {} pull
\end{verbatim}

Another example is to \texttt{make check} all \texttt{github.com} repositories from user \texttt{pbizopoulos}:
\begin{verbatim}
find $HOME/github.com/pbizopoulos/ -name Makefile |\
xargs dirname |\
xargs -I{} -P0 make check -C {}
\end{verbatim}

\section{Workflow}

\begin{algorithm}
	\KwData{New user/developer requirement}
	\If{requirement is unimportant}{
		\Return{}
	}
	\If{it can be solved without software}{
		\Return{}
	}
	\If{a package already exists}{
		Introduce the package as a dependency in the relevant main.py\;
		\Return{}
	}
	\If{a relevant repository exists}{
		Make repository compatible with methodology\;
		\Return{}
	}
	\If{it can be programmed within a relevant main.py}{
		Program it in a relevant main.py\;
		\Return{}
	}
	\If{it can be put as a parent/child of another directory}{
		Create a new parent/child directory of another directory\;
		\Return{}
	}
	Create a new repository\;
	\Return{}
	\caption{Workflow for new user/developer requirement}
\end{algorithm}

\bibliographystyle{IEEEtran}
\bibliography{ms.bib}

\end{document}
