\documentclass[journal]{IEEEtran}
\usepackage[nopdftrailerid=true]{pdfprivacy}

\begin{document}

\title{Principle of least action as a software development method}

\author{Paschalis~Bizopoulos
\thanks{P. Bizopoulos is an Independent Researcher, Thessaloniki, Greece e-mail: pbizopoulos@protonmail.com}}

\maketitle

\begin{abstract}
	This paper proposes the principle of least action as a software development method.
	We release open source repositories of the examples use cases of this method.
\end{abstract}

\section{Introduction}
Action is this.
Principle of least action (PLA) is that.
Software development practices like agile~\cite{abrahamsson2017agile}.

PLA can be used throughout all software-related decisions:
\begin{enumerate}
	\item whether to use dependencies (e.g. libraries, frameworks) or implement a build-in solution
	\item whether to use standards or extend/restrict/ignore them
	\item whether to implement/drop a feature
	\item whether to use a tool (e.g. text editor, IDE)
\end{enumerate}

PLA can be used for both the user and the developer across two axis: space (other machines) and time
a. characteristics of the user (e.g. technical capabilities)
b. characteristics of the user environment (e.g. Operating system)
c. characteristics of the developer (e.g. technical capabilities)
choose the technologies that allow the developer to minimize the development and user path length through space (other computers) and time (in the future)
d. characteristics of the developer environment (e.g. Operating system)

Action is relative to the observer.
The developer sees the path that corresponds to the development procedure while the user sees the path that corresponds the usage of the binary
both developers and users want to reduce their relative action. A minimum is achieved at a point
moreover the cost and benefits are relative to the user and developer respectively
if the benefit is greater than the cost then it is implemented
the benefit of achieving a target is balanced with the cost of the time required to achieve it

The aim of this paper is translating the idea of action and PLA to software development and then show the implementation that uses Makefile+containers.
present the Makefile for python and tex for document generation and then the Makefile for website creation
creating an application (the target) sometimes requires creating tools (as intermediate targets) that can be use to reduce the development time (the path length to the target)
it is not just about minimizing time or being completely accurate in satisfying the requirements. A combination of these is required
the path to the target is the source code which can be evaluated with make check. The path should fulfill the needs of the developers and can change
the target is the application which can be reached with make. The target should fulfill the needs of the users and can change

\section{Development environment/tools}
Mention install-archlinux.
Mention vim etc.

\section{The Makefile}
PLA has been applied in the following aspects of the Makefile:
\begin{itemize}
	\item this
	\item that
\end{itemize}

consistency across time from the developer perspective means that the technology used to program it will be available in the future
consistency across time from the user perspective practically means that the application should be finished from the point of view of features and only bug fixes might be developed
makes use of pre-existing standards: GNU standard for Makefile conventions and POSIX for the use of UNIX tools
the portability of the Makefile is achieved by using the mandatory UNIX utilities also mentioned by the GNU standard except for docker where needed
there is an inverse relationship between consistency and configurability
think about that subsets of standards have been used
time includes user interation to reach the target and total development time (including maintainance)
users care about the application, developers care about the code
we define a target of reducing the time required to create and navigate windows in a computing environment
we use Makefile to create the isolated environment (container) by bootstraping all required files if not already present and to make sure that subsequent changes to the requirements update the environment. The DAG should be as project agnostic as possible thus sentinel files are used
we use these specific targets for the Makefile (all, check, clean, help) because the first three are standard targets recommended by GNU standard targets and the fourth one is used in the linux kernel
The Makefile is only responsible for install the requirements inside the container.
The specific command arguments that e.g. check applications take should be done outside of the container to prevent the cache invalidation of the container.

\section{Various}
semantic html is a standard. And standards should be used if no personal opinion is required in the subject.

\section{Variable naming}
think prime-modifier-class naming scheme and how it relates to the action-as-information

\section{Example use case: technical document}
This section provides a use case of the Makefile and also serves as a manual for developing using this method.
Making use of the \textit{datatool} package the values of the following variables are not directly referred in the main \textit{.tex} file but they are instead read by an intermediate \textit{.tex} file created by the \textit{results code}.
Additionally, making use of random seed setting we consistenly get the same figures.
We also use the \textit{to\_latex} command which automatically converts a dataframe table to a \LaTeX\ table.

\section{About the use of Make}
The proposed Makefile is based on plain text file editing and can be used with any operating system that supports containers and Make.
Make has existed for decades and has passed the test of time, while the container technology was standardized with the Open Container Initiative and there exist alternatives such as Podman that could be used as drop-in replacements for Docker.
Moreover the Makefile can be combined with any version control system, container registry provider and text editor thus preventing `app/vendor lock-in' situations.

Use cases of the Makefile include:
\begin{itemize}
	\item regression testing and debugging, to ensure that changes to \textit{code} do not alter \textit{results},
	\item common development environment across multiple \textit{authors},
	\item coauthors, reviewers, journal editors or other researchers can reproduce the \textit{document} with few requirements.
\end{itemize}

\section*{Conclusion}

\bibliographystyle{IEEEtran}
\bibliography{ms.bib}

\end{document}
